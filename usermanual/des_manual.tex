This is a user manual for the desktop client. It will provide guides on how to use the client and the different functionalities it holds. The screenshots shown in this document are made from a Linux machine, but the Desktop application also runs on Windows or Mac, and will follow the design priciples thereafter. Because of this, the look of the client may vary, but the functionality is the same, and hopefully this manual will be of service anyways. 

\subsection{Login and startup}
When you start this application the first thing that's displayed is a login screen, as illustrated in \refer{fig:des_login-pic}. In this screen you enter your username, password and the IP-Address for the server and then press login to enter the Genomizer Desktop.

\begin{figure}[htb]
	\addScaledImage{0.5}{des_loginpicture.png}
	\caption{Screenshot of the login screen.}
	\label{fig:des_login-pic}
\end{figure}
The application is built with tabs, as illustrated below in the upper part of \refer{fig:des_desktop-view}. Each tab contains separate features of the application. There are six tabs: Search, Upload, Process, Workspace, Analyze and Administration.
\begin{figure}[htb]
	\addImage{des_searchtab.png}
	\caption{Illustration of the different tabs of Genomizer Desktop and displaying the Search tab.}
	\label{fig:des_desktop-view}
\end{figure}
\FloatBarrier

\subsection{Search}
The first tab that meets the user after logging in is the Search tab, illustrated in \refer{fig:des_desktop-view}. The Search tab uses the same query building technique as the “Pubmed Advanced Search Builder”. It has one text field where you either can type in the query yourself or you can use the query builder below it. Each row in the query builder has at most five components. These are a logical expression, an annotation name field, a free text field or a drop down menu to insert search words, a minus button and a plus button. The minus button removes a row and the plus button adds a row. These buttons are however not available in each row. The plus button is only available in the last row. The minus button is available in every row except if there is only one row in the query builder. The logical expressions combines the annotations, so they are available in every row but the first.
By writing in the annotation text field or selecting a value in the drop down menu you can specify the query the row will produce. Together each row builds a full query. As illustrated in \refer{fig:des_search-query} below.
\begin{figure}[htb]
	\addImage{des_searchtab.png}
	\caption{Illustration of a query, made by the query builder.}
	\label{fig:des_search-query}
\end{figure}
\FloatBarrier

\subsection{Upload}
If the user needs to upload a file to the database it can be done through the upload tab.
When the tab is pressed the user gets presented with a text field and a button where they can search for an existing experiment to upload to and another button for adding a new experiment. When the user presses the new experiment button, the user is presented with annotations they can choose between. If they pressed the existing experiment button, they are instead presented with annotations they can't choose between. They are also presented with buttons named Select files and Upload selected files, with which the user can choose a file via a file browser and upload them. The upload tab is illustrated in \refer{fig:des_upload-view} and the file browser is illustrated in \refer{fig:des_upload}.
\begin{figure}[htb]
	\addImage{des_uploadexisting.png}
	\caption{Illustration of the upload tab where the browse and upload functions are shown.}
	\label{fig:des_upload-view}
\end{figure}

\begin{figure}[htb]
	\addScaledImage{0.4}{des_uploadselect.png}
	\caption{Choosing a file for uploading}
	\label{fig:des_upload}
\end{figure}
\FloatBarrier

\subsection{Process}
In the process tab there is a list called Files that on the left side of the tab. From this list the user can mark RAW-files and choose to create profile data. By left clicking on the files they will be marked. If the user left clicks once again on the same file it will be unmarked. If the user then presses the Create profile data button which is visable in the middle of the tab see \refer{fig:des_process-view}, all the files that are marked will now be processed to profile data. This list of files will be empty unless the user has chosen to process selected RAW-files from the workspace tab. If that is the case then those selected RAW-files will then be visable in the list of files in the process tab. When the user has selected some RAW files the user has the option to change conversion parameters that is above the create profile data button as illustrated in \refer{fig:des_process-view}. These parameters has pre-set values. The conversion parameters are Flags, Genome release files, Window size,Smooth type,Step position,Step size,Print mean and Print zeros. If the user has selected some RAW-files and pressed the Create profile button, then if all went well and the server could convert the files a message "The server has converted: filename" will print in Convert Files for each file that was converted to profile data. If for some reason the server couldn't create profile data for any RAW-file another message "WARNING - The server couldn't convert: filename" will print in Convert Files that is visible in the middle bottom of the process tab see \refer{fig:des_process-view}.


\begin{figure}[htb]
	\addImage{des_processtab.png}
	\caption{Screenshot of the process tab in the program.}
	\label{fig:des_process-view}
\end{figure}
\FloatBarrier

\subsection{Workspace}
The Workspace View is the heart of the application for the user. Chosen files from the search is collected in this view, and the user can choose different options to process and analyze the data. 
For now, the only function here that is implemented is Download, but more will come.
\subsubsection{Download}
The user can make the choice to download. If the user presses the DOWNLOAD SELECTED button seen in \refer{fig:des_workspace-view}, then a pop up menu will present it self to the user as illustrated in \refer{fig:des_download-view}. In the pop up menu, the files that were selected in the workspace will be shown, then the user can choose a file format for each of the files (this functionality is not yet implemented). If the download button is pressed, the user gets to choose a directory where the files will be saved. When a directory has been chosen, the files get downloaded and saved in folders named after the experiments in which they belong. 
\begin{figure}[htb]
	\addImage{des_workspaceselect.png}
	\caption{Screenshot of the workspace tab in the program.}
	\label{fig:des_workspace-view}
\end{figure}
\begin{figure}[htb]
	\addImage{des_download.png}
	\caption{The download files pop up menu}
	\label{fig:des_download-view}
\end{figure}
\FloatBarrier

\subsection{Administration}
The system administration tools for the desktop client is available under the Administration tab. When a user selects the add button in the sidepanel a new popup windows appears. It is possible to write the name of the new annotation and name of new categories in this popup, as well as check a forced annotation box. See \refer{fig:adm_desktopgui}.
\begin{figure}[h!]
\addImage{des_addAnnotation.png}
\caption{The add new annotation popup.}
\label{fig:adm_desktopgui}
\end{figure}
