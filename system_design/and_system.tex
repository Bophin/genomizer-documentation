This section describes the architectural design of the Android application. All coded functionality is described in this section, inluding the functionality that is not fully integrated. 
\subsubsection{Class descriptions}\label{sec:and_classdescription}
This section focuses on the functionality of each class.

The connection between the classes labeled model can be seen in \refer{fig:and_umlmodel} while the other classes can be seen in \refer{fig:and_uml}
	\begin{figure}[h]
		\addImage{and_UML_model_sprint2.jpg}
		\caption{Android UML of model}
		\label{fig:and_umlmodel}
	\end{figure}  
\begin{figure}[h]	\addImageVertical{and_UML_android_sprint2.jpg}
		\caption{Android UML without model}
		\label{fig:and_uml}
	\end{figure}  
    \FloatBarrier
\subsubsection{Android activities} 
Activities are used as frames to display various fragments. For anything to be displayed it must be a part of a fragment that is a part of an activity. Each activity used in this application extends the class SingleFragmentActivity which has the purpose of creating a new fragment of an arbitrary type that displays a single fragment. This means that in its current state the application contains one fragment per activity and each fragment described in this section is coupled with an activity
\subsubsection{Fragment Class: LoginFragment}
This fragment will be the first view the user sees when the application is started, it allows the user to specify username and password. ComHandler \ref{sec:and_class_comhandler} is then used to send and validate the login request. If the username and password are incorrect the user will be informed through an android toast explaining what happened. Likewise if the server cannot be reached.

A successful login will start the SearchListFragment \ref{sec:and_class_search}.
\subsubsection{Fragment Class: SearchListFragment}\label{sec:and_class_search}
Handles the search-view for the \appName\ app, the annotations that can be chosen are downloaded from the server and they are, as such, dynamic.
\subsubsection{Fragment Class: ExperimentListFragment}
This fragment handles the displaying of search results to the user. The fragment includes a ListView with an ArrayAdapter set to it. An OnItemClickListener is used to detect when the user is selecting an item in the list and is currently starting FileListActivity \ref{sec:and_class_filelist} when a list item is selected. This fragment receives a HashMap with search values from SearchListFragment \ref{sec:and_class_search} and when activity is starting an ASyncTask is started to send and receive search results from the server through the ComHandler \ref{sec:and_class_comhandler} class. When an experiment is selected from the list the file names belonging to that experiment is sent to FileListFragment \ref{sec:and_class_filelist} that will display the file information. 
\subsubsection{Fragment Class: FileListFragment}\label{sec:and_class_filelist}
This fragment displays all files associated with a chosen experiment. The fragment is using three ListViews, one for each data type. The data types that are available are: raw, region and profile. Each list element will show the name of the data file and have a checkbox connected to it. A custom ArrayAdapter is used to handle checkbox interaction and displaying information to the user. There is option to select multiple files in the view by checking several checkboxes. The file names displayed are the ones currently available from the server for each available experiment.
\subsubsection{Model Class: ComHandler}\label{sec:and_class_comhandler}
Is a static object that is called by the fragments in the view to gain access to the models different functions. At this stage the ComHandler can be used to login, search for files and to request raw to profile conversions, although the latter is not yet integrated. The url that ComHandler tries to communicate with can be changed with a public method which makes it possible to implement a way for the user to change server.
\subsubsection{Model Class: Communicator}
Intended to manage the sending and receiving of messages between the server and the application using a http connection.
\subsubsection{Model Class: MsgFactory}
Creates the JSON messages that can be sent to the server.
\subsubsection{Model Class: MessageDeconstructor}
Interprets JSON messages and returns appropriate information.
\subsubsection{Model Class: GenomizerHttpPackage}
Used to store the body and status code of an http-response.
\subsubsection{Model Class: GeneFile}
Used to store and transfer the information of a genome file.
\subsubsection{Model Class: Annotation}
Used to store and transfer one or several annotations and their value.
\subsubsection{Model Class: Experiment}
Used to store and transfer information about an experiment.
\subsubsection{Model Class: ProcessingParameters}
Used to simplify the transfer of parameters in a processing request.
