Our system for the file storage is basically built up with a database and a file system, where the header information of a file is stored in the database and the real file is stored in the filesystem. A filepath string is stored in the database as “path”, so the user can find where the actual file is stored in the filesystem. You can see the database design in the schema below (\refer{fig:dat_databaseSchema} ). \\

Note that this section only describes the advanced system design that needs further explanation than just reading the code. Also note that it exists some javadocumentation that explains all methods in each class further deeper and more specific than this documentation that might be useful for those that will continue developing this system. Those can be found in the same file directory as the program, in the sub dir "/doc".

\begin{figure}[htb]
\addImage{dat_schema_V3.jpg}
\caption{Schema for the database.}
\label{fig:dat_databaseSchema}
\end{figure}

\underline{Comments about the database design:}

* FileID is a unique number for a specific file. It will be autogenerated by the database when you insert one file row.\\
* Path is where the actual file is stored in the filesystem, example:

\centerline{$home/data/experiment_1/raw/rawFile1.raw$}

Since many files are stored in pairs, the InputFilePath is the path to where the other file pair  is stored.\\
* MetaData is the arguments used when converting. \\
* IsPrivate is a boolean (T or F) for making the file private or not.\\
* Role in User Info determines the user rights for one user in the system, examples could be admin, user, etc.\\
"Annotated With" is the table that connects one experiment to one annotation, example:
\begin{center}
  \begin{tabular}{| l | l | l | l|}
    \hline
    $Experiment_1$ & Species & Dog\\ \hline
  \end{tabular}
\end{center}
* "Annotation" is the table containing the annotation with a preset default value, but not any choices. example:
\begin{center}
  \begin{tabular}{| l | l | l | l|}
    \hline
    Species & DropDown & Human & T \\ \hline
  \end{tabular}
\end{center}
"Annotation choices" is the actual values for a specific annotation, example:\\
\begin{center}
  \begin{tabular}{| l | l | l | l|}
    \hline
    Species & Dog \\ \hline
    Species & Fly \\ \hline
  \end{tabular}
\end{center}

\subsubsection{Methods description}
Below is some interaction processes described and what happens in the \term{database\-Accessor} class that needs further description than just viewing the class or the javaDoc.\\
\\
\underline{GetExperiment:} Will find a specific experiment from one experimentID, from the database and will return an experiment object for that experiment. The object will contain information about the experiment id, experiment annotations connected to that experiment, and all files connected to that experiment with their headers. The object contains both setter and getter methods. \\
\\
\underline{Adding one experiment step by step:}
There are some steps that need to be done in the right order to be able to add one experiment into the database. The order is as follows:\\
\\
1: First you will need to call the "addExperiment" method. It will add one experiment to the database witout any annotations set to that experiment. 

2: Then you add the annotations in general that should exist in the database.This can be done in many different methods depending on what purpose the annotation should have. Those are:\\
\\
a) AddFreeTextAnnotation: adds a free text annotation that will not be visible as a dropdown choice for the gui. Example for one entry line in the database could be: "Tissue,FreeText,Null,T".\\
\\
b) AddDropDownAnnotation: adds one annotation that will be visible as a dropdown choice for the gui. This method wants the label and a list of all annotation choices that the label should have, and also a default value. This method will fail if that annotation already exists in the database. Example of one entry could be: "Sex, DropDown, Male, T".\\
\\
c) AddDropDownAnnotationValue: If the method above fail because that annotation already exist in the database, you can run this method to just add one annotation choice to one existing annotation label, example: "species,dog".\\
\\
3: Once the annotation is set, you can connect the experiment with the annotations. This is done by calling the method "annotateExperiment". It is also here the FreeText annotations stores its values.\\
\\
Now the experiment should be in the database with annotation values, and is now ready to have files added to it.\\
\\
\underline{Changes to Annotations:} The possible methods for changing existing annotations are:\\
\\
* ChangeAnnotationLabel: changes one entire label in the database.\\Example: "Sex, Tissue, etc.".\\
\\
* ChangeAnnotationValue: change one value for a specific annotation label.\\Example: "male,female,fly,etc.".\\
\\
* GetChoices: Gets you all the available annotation choices connected to one label that you send in as inparameter. Example is "sex" that might return a list with values "Male,Female,Unisex,Unknown".\\
\\
* GetAnnotations: returns all annotation labels currently stored in the database. Examples could be "Sex,Species,Tissue,etc.".\\
\\
* GetAllAnnotationObjects: Does the same as the method GetAnnotation, but return it as an arrayList instead.\\
\\
* GetAnnotationObject: one method where you send in one label and get back one annotation object containing the "annotation" row  from the database, aswell as all its connected annotation choices.\\
\\
* GetAnnotationObjects: Same as getAnnotationObject but it returns a list of annotation objects from a list of annotation labels.\\
\\
* DeleteAnnotation: Deletes an entire annotation from the database. Since the database is configured to delete on cascade, all annotation choices connected to that annotation label will be removed, also the connection to the experiments with that annotation label.\\
\\
* RemoveAnnotationValue: Removes a single annotation value connected to a label, for example: "fly", or "arm".\\
\newpage
\underline{Adding files:} To add a file you will need to have an experiment added before you call the "addNewFile" method. Some files uses multiple files like raw data so make sure that you upload them together and that the "InputFilePath" points to their other file pair.\\
\\
\underline{Removing files:} Also here, if you delete one file that comes as an file pair, you must also delete the other file thorugh this method.\\

\underline{Other:} There exist other methods as well, but these does not need any further description then what is written in the java class DataBaseAccessor. Down bellow follows a more overall description of each class that is needed in the datastorage part of the system.\\
