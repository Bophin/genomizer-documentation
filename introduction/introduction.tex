% When conducting experiments and analyzing data within the epigenetics field, a large amount of data is generated. This data needs to be handled and stored in an easy, secure and efficient way. Analyzing this data is today a very time consuming task, involving several manual steps with little to none user friendliness. These steps might also change and/or be replaced. A system for analyzing and storing epigenetics data safely and efficiently in a single application does not exist today. 

% Within the PVT Project we are developing a system that addresses these issues. The system is built for experienced epigenetic scientists as well as for new users. This system focuses on making analyzing and storing epigenetics data less time consuming and more efficient by gathering these tasks in one common application and by using a database for storing. The system will help users to easily search the database for data. Data from the database could either be used for analyzing just by marking what data to be used or to be downloaded to a local space in a wide variety of different file formats. The database is accessed from multiple devices (Windows, Linux, OSX, Android and iPhone) using separate client applications. 

\appName\ is a system for storing and analyzing \term{DNA}-sequences. It was designed for researchers in the field of epigenetics, who are interested in where on a \term{DNA} string a certain protein bind. In order to get this information, experiments are conducted and \term{raw} data files collected. These data files are then converted, in a series of steps, to files suitable for analysis. \appName\ allows the researchers to upload \term{raw} files to a server and automate the generation of analysis data. 

\appName\ is developed by the students of a course in software engineering at \term{Umeå University}. This documentation is directed towards three main groups. The first group are the users that wants to use the software. The second group are the system administrators that wants to maintain the system and help other users with more advanced tasks. The third and final group are system developers that wishes to expand and improve the current software.

The first part of this documentation describes the target group of the software. This is followed by instructions of how to use the software for the common user. Then a chapter follows with instructions of how to deploy the server together with the software. This chapter is directed towards administrators and developers. After these initial chapters that is focused on using the system follows chapters with more indepth information on how the software is designed and implemented. 

% Finally at the end of the documentation are appendixes that helps describe different parts of the software and the software development.