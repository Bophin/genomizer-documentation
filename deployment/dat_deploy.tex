The following guide assumes access to a server with postgresql installed. If you do not yet have a database, username and password for \appName\ to use proceed to \emph{Set up postgresql account}.

    This guide was written using Ubuntu 14.04 LTS (GNU/Linux 3.13.0-24-generic i686) and postgres-9.3.4.

    \subsubsection{Set up postgresql account}
      This step is only required if you do not already have a \texttt{psql} username and password. If you have been assigned this from a sysadmin proceed to \emph{Upload SQL Script to server}.

    \begin{enumerate}
      \item Log in to the server:
      \begin{verbatim}
> ssh <username>@<host>
      \end{verbatim}

      \item Become sudo-user “postgres”:
      \begin{verbatim}
> sudo su postgres
      \end{verbatim}

      \item Add yourself as a postgresql user:
      \begin{verbatim}
> createuser <username>
      \end{verbatim}

      \item Log into postgresql as root:
      \begin{verbatim}
> psql
      \end{verbatim}

      \item Set your password:
      \begin{verbatim}
> \password <username>
      \end{verbatim}

      \item Create database:
      \begin{verbatim}
> create database genomizer;
      \end{verbatim}

      \item Grant yourself all permissions on the genomizer database: \begin{verbatim}
> grant all on database genomizer to <username>;
> \q
      \end{verbatim}

      \item Navigate to postgresql configuration folder:
      \begin{verbatim}
> cd /
> cd etc/postgresql/9.3/main
      \end{verbatim}

      \item Navigate to postgresql configuration folder:
      \begin{verbatim}
> sudo nano postgresql.conf
      \end{verbatim}

      \item Change connection settings:\\Locate line: 
      \begin{verbatim}
#listen_adresses = ‘<settings>’    # what IP address(es) to listen on;
      \end{verbatim}
      Change to:
      \begin{verbatim}
listen_addresses = '*'    # what IP address(es) to listen on;
      \end{verbatim}

      \item Write changes and exit:\\
      Hold down ctrl and press o\\
      Hold down ctrl and press x

      \item Open configuration file:
      \begin{verbatim}
> sudo nano pg_hba.conf
      \end{verbatim}

      \item Change Client Authentication Configuration:\\Locate the heading: 
      \begin{verbatim}
# IPv4 local connections:
      \end{verbatim}
      Under the heading, add the line:
      \begin{verbatim}
host    all    all    127.0.0.1/32    md5
      \end{verbatim}

      \item Write changes and exit:\\
      Hold down ctrl and press o\\
      Hold down ctrl and press x

      \item Restart postgresql:
      \begin{verbatim}
> cd /
> sudo /etc/init.d/postgresql restart
      \end{verbatim}

    \end{enumerate}

    \subsubsection{Upload SQL Script to server}

    \begin{enumerate}

      \item In a termainal window navigate to the folder where the \verb+genomizer_database_tables.sql+ script resides.

      \item Establish secure ftp connection to the server:
      \begin{verbatim}
> sftp <username>@<host>
      \end{verbatim}

      \item Create a new folder on the server:
      \begin{verbatim}
> mkdir SqlScripts
      \end{verbatim}

      \item Upload \verb+genomizer_database_tables.sql+:
      \begin{verbatim}
> put genomizer_database_tables.sql SqlScripts/
      \end{verbatim}

      \item Exit \texttt{sftp}:
      \begin{verbatim}
> exit
      \end{verbatim}

    \end{enumerate}

    \subsubsection{Create the \appName\ Tables}

    \begin{enumerate}

      \item Log in to the server:
      \begin{verbatim}
> ssh <username>@<host>
      \end{verbatim}

      \item Log in to the database:
      \begin{verbatim}
> psql genomizer
      \end{verbatim}

      \item Run \verb+genomizer_database_tables.sql+
      \begin{verbatim}
> \i SqlScripts/genomizer_database_tables.sql
      \end{verbatim}


    \end{enumerate}

The \appName\ database is now ready to use.
